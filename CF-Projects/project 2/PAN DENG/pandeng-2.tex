\documentclass{article}
\usepackage[utf8]{inputenc}

\title{OptionPrice}
\author{pd90506 }
\date{May 2014}

\begin{document}

\maketitle

\section{Determination of the option price of ZNJS}

Here we want to first determine the european call option price, since by B-S formula
\begin{equation}
C(S,t) = SN(d_1)-Ee^{-r(T-t)}N(d_2)
\end{equation}
where
\[
    d_1 = \frac{\ln{S/E}+(r+\frac{1}{2} \sigma^2)(T-t)}{\sigma \sqrt{T-t}}
\]
\[
    d_2 = \frac{\ln{S/E}+(r-\frac{1}{2} \sigma^2)(T-t)}{\sigma \sqrt{T-t}}
\]
Since S, E, T, t, and r are all known, the only unknown parameter is the volatility $\sigma$.

Then how to evaluate the volatility from the historical data? 

Suppose that we have the stock price data every trading day and denote as $S_i$.
Further assume that the stock price follows the geometric Brownian motion, that is
\begin{equation}
S_{i+1} = S_i\exp\{(\mu -\frac12 \sigma^2)(t_{i+1}-t_i) + \sigma \sqrt{t_{i+1}-t_i}Z_i\}
\end{equation}
where $Z_i$ follows standard normal distribution.

Take the log of both side, here we have
\begin{equation}
\ln{S_{i+1}}-\ln{S_i} = (\mu -\frac12 \sigma^2)(t_{i+1}-t_i) + \sigma \sqrt{t_{i+1}-t_i}Z_i
\end{equation}

Now let's do some simplifications, denote $D_i = \ln{S_{i+1}}-\ln{S_i}$ and $\Delta t =t_{i+1}-t_i $ then the equation becomes
\begin{equation}
D_i = (\mu -\frac12 \sigma^2)\Delta t + \sigma \sqrt{\Delta t}Z_i
\end{equation}

We can evaluate the variance of $D_i$ using MatLab, say $d^2$, and the corresponding evaluated volatility of $S_i$ must be
\begin{equation}
\hat{\sigma}^2 = \frac{d^2}{\Delta t}
\end{equation}

\section{Evaluation}
For simplicity, we claim there are 250 trading days per year. And the data we use is taken daily, so the quantity $\Delta t = 1/250$ presumed. To assure the consistency, we only use the open price of each day.

We get the variance $d^2 = 9.0877\times 10^{-4}$, then $\hat{\sigma}^2 = 0.2272$, and then $\hat{\sigma} = 0.4766$. Then substitute it into the pricing formula we can get the answer.

\section{Application}
Today's stock price $S = 6.96$ , take the interest rate of 10-year treasury bonds as the risk-free interest rate, $r = 4.1348\%$, if we want to sell call options expire in 3 months, with strike price 8 yuan, then the option price can be calculated:
\[
    d_1 = \frac{\ln{S/E}+(r+\frac{1}{2} \sigma^2)(T-t)}{\sigma \sqrt{T-t}} = \frac{\ln{6.96/8}+(0.0413+\frac{1}{2} 0.2272)3/12}{0.4766 \sqrt{3/12}}=-0.4219
\]
\[
    d_2 = \frac{\ln{S/E}+(r-\frac{1}{2} \sigma^2)(T-t)}{\sigma \sqrt{T-t}} = \frac{\ln{6.96/8}+(0.0413-\frac{1}{2} 0.2272)3/12}{0.4766 \sqrt{3/12}}= -0.6602
\]
and then
\[
    C(S,t) = SN(d_1)-Ee^{-r(T-t)}N(d_2) = 6.96N(-0.4219)-8\times e^{-0.0413\times 3/12} N(-0.6602) = 0.3268
\]
\end{document}
